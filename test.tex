
\deleted[id=FR]{%What is groundwater management and why is it important
Groundwater management plays a key role for sustainable use of groundwater resources. As cities and their populations grow, their demand for water increases accordingly. In response, cities are drilling new groundwater wells to keep up with demand. Groundwater resources often are dependent on surface-water features such as nearby rivers. Excessive groundwater pumping can reduce the flow of nearby rivers and affect those who rely on these rivers for their water supply. Groundwater management involves a set of methods that prevent excessive use of groundwater resources while still being able to satisfy water demands.} %Groundwater management is important because prevents water demands from causing irreparable damage to precious groundwater resources.

\deleted[id=FR]{%OLD% What is the Well Placement Problem (WPP)?
A key task in groundwater management is well-field design, which involves determining how to place and build a set of pumping wells and to optimize the use of water resources. A subproblem within well-field design is the well placement problem. Given the groundwater resource and a set of constraints such as the number of wells, the challenge is to determine the best places to install the wells. A solution for the well placement problem can help cities and industries design well fields that can extract groundwater economically and efficiently. Unfortunately, determining the optimal placement of wells is a challenging task. The optimal configuration of a well field is dependent on numerous factors that range from the economics of installing wells to the hydraulic properties of the aquifer. From a mathematical standpoint, the problem involves maximizing a non-linear, multi-modal objective function. Such a task is computationally expensive because of the large size and complexity of numerical flow\deleted[id=FR]{ and transport} models used to simulate the performance of these well fields under the stress conditions.}

\added[id=FR]{
%What is the Well Placement Problem (WPP)?
Well placement design refers to the challenge of finding the optimal locations to install a set of wells. This is a common problem found in many fields of natural resource management. In the petroleum industry, solving the well placement problem allows the design of optimal well fields that can efficiently and economically produce from hydrocarbon reservoirs \citep{sarma2008efficient,feng2012optimizing,nwankwor2013hybrid,nozohour2016application}. For water resource management, solving the well placement problem can lead to efficient well field designs for producing groundwater or for groundwater remediation \citep{park2004multi,bayer2009optimized,elcci2014differential,wang1994optimal}. Finding a good optimization method for well placement design allows the construction of more efficient, effective and economic well fields.}

\added[id=FR]{
%What is the Well Placement Problem (WPP)?
Well placement design refers to the challenge of finding the optimal locations to install a set of wells. This is a common problem found in many fields of natural resource management. In the petroleum industry, solving the well placement problem allows the design of optimal well fields that can efficiently and economically produce from hydrocarbon reservoirs \citep{sarma2008efficient,feng2012optimizing,nwankwor2013hybrid,nozohour2016application}. For water resource management, solving the well placement problem can lead to efficient well field designs for producing groundwater or for groundwater remediation \citep{park2004multi,bayer2009optimized,elcci2014differential,wang1994optimal}. Finding a good optimization method for well placement design allows the construction of more efficient, effective and economic well fields.}

\deleted[id=FR]{%%What are popular methods to solve the WPP (DE, PSO)
In previous decades, many algorithms have been developed to solve the well placement problem. Because calculating the absolutely best well-field configuration is a computationally intractable problem, most solutions rely on stochastic and heuristic methods. Popular algorithms in the literature include particle swarm optimization and differential evolution citep{minton2012comparison}.}